 %% Copyright (C) 2011, Andrea Cimino, All Rights Reserved.
 %% This file is distributed under the terms of the Creative Commons
 %% Licence Non-Commercial Share-Alike license


%% Useful stuff for separate compilation.
\ifx\ismaindoc\undefined
\providecommand{\inbpdocument}{
 \documentclass[11pt,a4paper,twoside,titlepage]{scrbook}
%%%%%%%%%%%%%%%%%%%%%%%%%%%%%%%%
%%%%%%%%%%% PACKAGES %%%%%%%%%%%
%%%%%%%%%%%%%%%%%%%%%%%%%%%%%%%%
% encoding
\usepackage[utf8x]{inputenc}
\usepackage[italian]{babel} % babel (suddivisione parole in sillabe)

\usepackage{amsfonts} % matematica
\usepackage{amsmath} % matematica
\usepackage{amssymb} % simboli vari
\usepackage{calrsfs}
\usepackage{caption}
\usepackage{enumerate}
\usepackage{extarrows} % matematica
\usepackage{keyval}
\usepackage{manfnt} % Simboli curva
\usepackage{mathtools} % matematica
\usepackage{multirow} 
\usepackage[usenames]{color} % colori con nome
\usepackage[pdftex]{graphicx}
\usepackage{epstopdf} % gestione file EPS
\usepackage{wrapfig} % per figure circondate da testo
\usepackage{framed}	% teoremi framed
\usepackage{fancyhdr} % header buffi
\usepackage[T1]{fontenc} % gestione hbox e vbox
\usepackage[a4paper]{geometry}
\usepackage{microtype} % gestione hbox e vbox
\usepackage[thref, amsthm, amsmath, framed]{ntheorem} % teoremi (avanzata)
%% \usepackage{prooftree} % gestione prof-tree
\usepackage{rotating}
\usepackage{stmaryrd}
\usepackage{subfig}
\usepackage{syntax} % syntattic stuff
\usepackage{txfonts}
\usepackage{verbatim} % migliorie al verbatim
\usepackage{hyperref}
%% \usepackage{qtree}
\usepackage{fancyvrb}
\usepackage{listings}
\usepackage{cancel}
\usepackage{tikz}
\usepackage{cclicenses}

\usepackage{bbding} %% Icons

%%%%%%%%%%%%%%%%%%%%%%%%%%%%%%%%
%%%%%%%%%%% GEOMETRY %%%%%%%%%%%
%%%%%%%%%%%%%%%%%%%%%%%%%%%%%%%%
\geometry{verbose,tmargin=2cm,bmargin=2.5cm,lmargin=2.5cm,rmargin=2cm}
\parindent0ex %% Remove paragraph indenting

%%%%%%%%%%%%%%%%%%%%%%%%%%%%%%%%
%%%%%%%%%%% CODE ENV %%%%%%%%%%%
%%%%%%%%%%%%%%%%%%%%%%%%%%%%%%%%
% codice
\newcounter{count}
\setcounter{count}{0}
\newenvironment{code}[1]
{
\color{lightgray}\hrulefill\color{code}
\stepcounter{count} {\bf\small Listato di codice \arabic{count}: {#1} }
\verbatim
}
{
\endverbatim
\color{lightgray}\hrulefill
\color{black}
\\
}

% codice semplice
\newenvironment{simplecode}
{
\color{code} \tt
}
{
\rm
}

 % Notation issues
\input{macros}

\makeatletter
\g@addto@macro\@verbatim\footnotesize
\makeatother



%%%%%%%%%%%%%%%%%%%%%%%%%%%%%%%%
%%%%%%%% THEOREMS FORMAT %%%%%%%
%%%%%%%%%%%%%%%%%%%%%%%%%%%%%%%%
% shaded theorems and proofs command
\definecolor{lightgray}{RGB}{230,230,230}
\def\theoremframecommand{\colorbox{lightgray}}

%%% theorems
\theoremstyle{break}
\theoremheaderfont{\normalfont\bfseries}
\theorembodyfont{\itshape}
\theoremsymbol{\ensuremath{\diamondsuit}}
\theoremseparator{\newline}
\newshadedtheorem{theo}{\includegraphics[scale=0.11]{imgs/book.png}
Teorema}[chapter]

%%% propositions
\theoremstyle{break}
\theoremheaderfont{\normalfont\bfseries}
\theorembodyfont{\itshape}
\theoremsymbol{\ensuremath{\diamondsuit}}
\theoremseparator{\newline}
\newshadedtheorem{proposition}{Proposizione}[chapter]

%%% exercises
\theoremstyle{break}
\theoremheaderfont{\normalfont\bfseries}
\theorembodyfont{\itshape}
\theoremsymbol{\ensuremath{\diamondsuit}}
\theoremseparator{\newline}
\newshadedtheorem{exercise}{Esercizio}[chapter]

%%% propositions
\theoremstyle{break}
\theoremheaderfont{\normalfont\bfseries}
\theorembodyfont{\itshape}
\theoremsymbol{\ensuremath{\diamondsuit}}
\theoremseparator{\newline}
\newshadedtheorem{property}{\PencilRightDown $\; $ Propriet\`a}[chapter]
%%% lemmas
\theoremstyle{break}
\theoremheaderfont{\normalfont\bfseries}
\theorembodyfont{\itshape}
\theoremsymbol{\ensuremath{\diamondsuit}}
\theoremseparator{\newline}
\newshadedtheorem{lemma}[theo]{Lemma}

%%% definitions
\theoremstyle{break}
\theoremsymbol{\ensuremath{\clubsuit}}
\theoremseparator{\newline}
\newshadedtheorem{defn}[theo]{Definizione}

%%% examples
\theoremstyle{break}
\theorembodyfont{\itshape}
\theoremsymbol{\ensuremath{\ast}}
\theoremseparator{\newline}
\newshadedtheorem{example}[theo]{Esempio}

%%% observations
\theoremstyle{break}
\theorembodyfont{\itshape}
\theoremsymbol{\ensuremath{\ast}}
\theoremseparator{\newline}
\newshadedtheorem{observation}[theo]{
\includegraphics[scale=0.06]{imgs/lens.png}
Osservazione
}

%%% notations
\newtheorem*{notaz}{Notazione}

%%% proofs
\newenvironment{thproof}
{
\vskip 0.03cm
\begin{small}
\textit{Dimostrazione. }
\color{code}
}
{
\color{black}
\end{small}
$ \square $
\vskip 0.2cm
}

%Notes
\newenvironment{notes}{%
  \def\FrameCommand{\colorbox{yellow}}%
  \MakeFramed {\FrameRestore}
\includegraphics[scale=0.02]{imgs/bulb.png}
 \textbf{Nota} \\
 }%
{\endMakeFramed}

%Work in progress
\newenvironment{workinprogress}{%
  \def\FrameCommand{\colorbox{pink}}%
 
 \MakeFramed 
 {\FrameRestore}
\lhdbend  \textbf{Work in progress} \\
\\
 }%
{\endMakeFramed}

%Openquestion
\newenvironment{openquestion}{%
  \def\FrameCommand{\colorbox{pink}}%
  \MakeFramed {\FrameRestore}
 \textbf{Domanda aperta} \\
 }%
{\endMakeFramed}

%TODO
\newenvironment{todo}{%
  \def\FrameCommand{\colorbox{pink}}%
  \MakeFramed {\FrameRestore}
 \textbf{TODO} \\
 }%
{\endMakeFramed}

%%%%%%%%%%%%%%%%%%%%%%%%%%%%%%%%
%%%%%%%%%%%% HEADER %%%%%%%%%%%%
%%%%%%%%%%%%%%%%%%%%%%%%%%%%%%%%
\pagestyle{fancy}
% i comandi seguenti impediscono la scrittura in maiuscolo
% dei nomi dei capitoli e dei paragrafi nelle intestazioni
\renewcommand{\chaptermark}[1]{\markboth{#1}{}}
\renewcommand{\sectionmark}[1]{\markright{\thesection\ #1}}
\fancyhf{} % rimuove l'attuale contenuto dell'intestazione
% e del pi\`e di pagina
\fancyhead[LE,RO]{\bfseries\thepage}
\fancyhead[LO]{\bfseries\rightmark}
\fancyhead[RE]{\bfseries\leftmark}
\renewcommand{\headrulewidth}{0.5pt}
\renewcommand{\footrulewidth}{0pt}
\addtolength{\headheight}{0.5pt} % riserva spazio per la linea
\fancypagestyle{plain}{%
\fancyhead{} % ignora, nello stile plain, le intestazioni
\renewcommand{\headrulewidth}{0pt} % e la linea
}


%%%%%%%%%%%%%%%%%%%%%%%%%%%%%%%%
%%%%%%%%%%%% COLORS %%%%%%%%%%%%
%%%%%%%%%%%%%%%%%%%%%%%%%%%%%%%%
\definecolor{code}{gray}{0.3}


%%%%%%%%%%%%%%%%%%%%%%%%%%%%%%%%
%%%%%%%%%%%% NUMBERS %%%%%%%%%%%
%%%%%%%%%%%%%%%%%%%%%%%%%%%%%%%%
\setcounter{tocdepth}{3}
\setcounter{secnumdepth}{3}


%%%%%%%%%%%%%%%%%%%%%%%%%%%%%%%%
%%%%%%%%%%% DOC DATA %%%%%%%%%%%
%%%%%%%%%%%%%%%%%%%%%%%%%%%%%%%%
\title{Appunti di MNO}
\author{Gruppo Informatici Rampanti}
\date{ott 2010 - mag 2011}

\pdfinfo{%
  /Title    (Appunti di MNO)
  /Author   (Andrea Cimino e Lorenzo Muti)
  /Creator  (Andrea Cimino)
  /Producer (Lorenzo Muti)
  /Subject  (MNO)
  /Keywords (MNO)
}


%%%%%%%%%%%%%%%%%%%%%%%%%%%%%%%%
%%%%%%%%%%%%% UTILS %%%%%%%%%%%%
%%%%%%%%%%%%%%%%%%%%%%%%%%%%%%%%
% binary symbols
\newcommand{\modder}{\vdash _{R}}

% vertical gaps
\newcommand{\askip}{\vspace{0.5cm}}
\newcommand{\bskip}{\vspace{1.0cm}}

% various symbols
\newcommand{\qedhere}{\ensuremath{\Box}}
\newcommand{\qed}{\hfill \ensuremath{\Box}}

% substitution
\newcommand{\subst}[2]{^{#1} / _{#2}}

% denotational semantics function names
\newcommand{\bbracket}[1]{\left\llbracket #1 \right\rrbracket}

\newcommand{\aexpr}{\mathcal{A}}
\newcommand{\bexpr}{\mathcal{B}}
\newcommand{\cexpr}{\mathcal{C}}
\newcommand{\Aexpr}[1]{\mathcal{A} \bbracket{#1}}
\newcommand{\Bexpr}[1]{\mathcal{B} \bbracket{#1}}
\newcommand{\Cexpr}[1]{\mathcal{C} \bbracket{#1}}

\newcommand{\semdomset}[1]{(V_{#1})_{\bot}}

% semantic evaluations
\newcommand{\opereval}[3]{\left\langle #1, #2 \right\rangle \rightarrow #3}
\newcommand{\denaeval}[3]{\Aexpr{#1} #2 = #3}
\newcommand{\denbeval}[3]{\Bexpr{#1} #2 = #3}
\newcommand{\denceval}[3]{\Cexpr{#1} #2 = #3}

% rotated sqsubseteqs
\newcommand{\upsqsubseteq}{ $\begin{rotate}{90} $\sqsubseteq$ \end{rotate}$ }
\newcommand{\downsqsubseteq}{ $\begin{rotate}{270} $\sqsubseteq$ \end{rotate}$ }

% Space after paragraph declaration
\makeatletter
\renewcommand\paragraph{\@startsection{paragraph}{4}{\z@}%
  {-3.25ex\@plus -1ex \@minus -.2ex}%
  {1.5ex \@plus .2ex}%
  {\normalfont\normalsize\bfseries}}
\makeatother

\begin{document}
}
\providecommand{\outbpdocument}{\end{document}}
\else
\providecommand{\inbpdocument}{}
\providecommand{\outbpdocument}{}
\fi



\inbpdocument 

%% 25 Marzo 2011
\chapter{Il problema nonlineare dei minimi quadrati}
\section{Minimi quadrati lineari}
Analizzeremo il problema dei minimi quadrati lineari dal punto di vista
dell'ottimizzazione. Siamo nel caso continuo non vincolato:
è un problema quindi che rientra nella classe di problemi
che abbiamo studiato. Analizziamone la forma:
$$ A \in \mathbb{R}^{m \times n}, b \in \mathbb{R}^{m} \qquad
 \min \{ || Ax -b ||_{2} : x \in \mathbb{R}^{n} \} $$
Per trattare il problema, riscriviamo la funzione obiettivo:
possiamo fare la seguente equivalenza che  agevoler\`a
meglio i conti (tale riscrittura vale solo nel caso lineare)
$$ \min \{ || Ax -b ||_{2} : x \in \mathbb{R}^{n} \} 
 \equiv
 \min \left\{ \frac{1}{2} || Ax -b ||_{2}^{2} : x \in \mathbb{R}^{n} \right\}
 $$
Abbiamo fatto una scalatura ed elevato al quadrato:
i minimi in tutte e due le scritture coincidono, in quanto
le funzioni che abbiamo applicato sono monotone.
Sviluppiamo la funzione obiettivo:

$$ f(x) = \frac{1}{2} || Ax - b ||_{2}^{2} = \frac{1}{2}(Ax-b)^{T}(Ax-b) =
\frac{1}{2}x^{T} A^{T}A x - (A^{T}b)^{T}x + \frac{1}{2} b^{T}b $$
Questa \`e una  funzione quadratica, di cui sappiamo calcolare immediatamente
il gradiente 
$$ \nabla f(x) = A^{T}A x - A^{T}b $$
mentre la matrice Hessiana \`e data da :
$$ \nabla^{2}f(x) = A^{T}A$$

\paragraph{Considerazione 1: annullamento del gradiente}
Dove si annulla il gradiente?
$$ \nabla f(x) = 0 \quad
 \Longleftrightarrow \quad 
A^{T}A x = A^{T}b $$
Quest'ultimo è il sistema delle equazioni normali: i punti stazionari
sono le soluzioni delle equazioni normali. \\
Sappiamo dalla teoria che si ha minimo se la funzione è \emph{convessa},
ed \`e noto inoltre che una funzione \`e convessa se la matrice Hessiana
($\nabla^{2}f(x)$) \`e semidefinita positiva.
Andiamo quindi a controllare se abbiamo questa condizione:
$$ x^{T} A^{T}A x = (Ax)^{T}Ax = || Ax||_{2}^{2} \geq 0$$
La matrice Hessiana \`e semidefinita positiva in ogni punto, quindi
la funzione è convessa: concludiamo quindi che i punti stazionari
coincidono con i punti di minimo. Abbiamo visto il problema dei minimi quadrati sotto un altro punto di vista, quello dell'ottimizzazione.

\paragraph{Considerazione 2: matrice A di rango massimo}
Se la matrice $A$ \`e di rango massimo,
abbiamo che la funzione \`e strettamente convessa: infatti
$$
\begin{array}{c}
 A \text { rango massimo } \quad
 \Longrightarrow  \quad ( Ax=0 \Longleftrightarrow x =0 ) \quad 
 \Longrightarrow \quad  x^{T}\nabla^{2} f(x)x  > 0 \; 
\forall x \neq 0  \\
\quad \Longrightarrow \quad f \text{ strettamente convessa }
\end{array}
$$
\`E noto inoltre che, quando la funzione \`e strettamente convessa, 
esiste un solo punto di minimo.

\paragraph{Considerazione 3}
$$ f(x) = \frac{1}{2} || Ax -b||_{2}^{2} =
\frac{1}{2}\left((A_1x -b_1)^{2} + (A_2x - b_2)^{2} +
\ldots   + (A_mx -b_m)^{2})\right)$$
$f(x)$ si dice lineare quando gli addendi $ (A_mx -b_m)$ sono lineari. \\
Questo \`e il punto di partenza per parlare del problema
dei minimi quadrati non lineare.

\section{Problema dei minimi quadrati non lineare}
Un problema dei minimi quadrati non lineare ha le seguenti caratteristiche:
$$ \min \{f(x) : x \in \mathbb{R}^{n} \} \quad
f(x) = \frac{1}{2} \displaystyle \sum_{j=1}^{m} r_j^{2}(x) $$
dove
$$r_j: \mathbb{R}^{n} \rightarrow \mathbb{R} \text{ non lineari } 
\qquad  x \rightarrow A_jx-b_j;$$
Un esempio di problema non lineare era il data fitting, visto nelle
prime lezioni. I metodi per risolvere questo problema li avremmo gi\`a ma,
poich\'e conosciamo la struttura della funzione obiettivo,
vogliamo sfruttarne la struttura  per specializzare i metodi che
 abbiamo visto fino ad adesso. \\
Supporremo inoltre $r_j$ differenziabile 2 volte.\\
Cercheremo di specializzare il metodo di Newton. \\
\paragraph{Calcolo del gradiente}
La funzione $r_j^{2}$ è la composizione di $r_j$ con una funzione $\varphi$
che associa ad un valore il suo quadrato
$$ r_j^2(x) = \varphi(r_j(x)) \quad \varphi: \mathbb{R} \rightarrow \mathbb{R}
\quad t \rightarrow t^2$$
Tale funzione è derivabile e vale
$$ \varphi'(t) = 2t$$
Il gradiente di una una funzione composta \`e noto per il teorema 
(\ref{theo:derivata-funzioni-composte}) : calcoliamolo in questo caso

\begin{equation}
\label{minqbigi:001}
\nabla r_j^{2} = \nabla \varphi(r_j(x)) =
\varphi'(r_j(x)) \nabla r_j(x) = 2 r_j(x) \nabla r_j(x)
\end{equation}


Quindi, usando la linearit\`a dei gradienti e utilizzando la
(\ref{minqbigi:001}) otteniamo
$$ \nabla f(x) = \frac{1}{2}
\displaystyle \sum_{j=1}^{m} \nabla r_j^{2}(x) = 
\displaystyle \sum_{j=1}^{m} r_j(x) \nabla r_j(x) = \ldots
$$
Cerchiamo una rappresentazione pi\`u compatta:
$$ R = (r_1, \ldots, r_m): \mathbb{R}^n \rightarrow \mathbb{R}^{m}$$
$$R(x) = (r_1(x), \ldots, r_m(x)) $$
Matrice Jacobiana, le cui righe sono i gradienti dei componenti
$$J_R(x) =
\left[
\begin{array}{c}
\nabla r_1(x)^{T}\\
\vdots  \\
\nabla r_m(x)^{T}
\end{array}
\right]
\in \mathbb{R}^{m \times n}
 $$

Il tutto si riduce quindi a:
$$ \nabla f(x) = \ldots = J_R(x)^{T}R(x)$$

\paragraph{Calcolo della matrice hessiana}
Calcoliamo le derivate seconde, la matrice hessiana di $f$
$$[ \nabla^{2}f(x)]_{kl} = 
\frac{\partial^{2} f(x)}{\partial x_k \partial x_l}=
\frac{\partial}{\partial x_k}\left(\frac{\partial f}{\partial x_l}(x)\right) =
\frac{\partial}{\partial x_k }
\left(\displaystyle \sum_{j=1}^{m} r_j(x) 
\frac{\partial r_j}{\partial x_l}(x) \right)
\underbracket{=}_{*)}
\displaystyle \sum_{j=1}^{m}
\underbracket{\frac{\partial r_j(x)}{\partial x_k} 
\frac{\partial r_j(x)}{\partial x_l}}_{[J_R(x)^{T}J_{R}(x)]_{kl}}
+ \displaystyle \sum_{j=1}^{m} r_j(x)
\underbracket{\frac{\partial^{2}r_j(x)}{\partial x_k \partial x_l}}_{[\nabla^{2} r_j(x)]_{kl}}
$$
\begin{notes}
 *) Si sta applicando la regola di derivazione del prodotto.
\end{notes}

$$
[J_R(x)^{T}J_{R}(x)]_{kl} = 
\left[
\begin{array}{ccc}
\nabla r_1(x)^{T}  & \cdots &  \nabla r_m(x)^{T} \\
\end{array}
\right]
\left[
\begin{array}{c}
\nabla r_1(x)^{T}\\
\vdots  \\
\nabla r_m(x)^{T}
\end{array}
\right]
$$\

In forma matriciale possiamo riscrivere tutto come
$$ \nabla^{2}f(x) = J_R(x)^{T}J_R(x) + 
 \displaystyle \sum_{j=1}^{m} r_j(x) \nabla^{2} r_j(x)$$
Poich\'e abbbiamo le formule esplicite del gradiente e della matrice
hessiana, possiamo sostituirle nei metodi visti
specializzarli.  Ad esempio nel metodo di Newton \ref{newtongradesatto}: $d^{k} \in
\mathbb{R}^{n}$ soluzione del sistema
$$ \nabla^{2}f(x^{k})d = - \nabla f(x^{k}) $$
Se $\nabla^{2} f(x^{k})$ è definita positiva, allora $d_{N}^{k}$ è una
direzione di discesa.  Cercheremo un'approssimazione del metodo di
Newton per togliere di mezzo i termini del secondo ordine: otterremo
il metodo di Gauss-Newton, per evitare il calcolo delle matrice
hessiane.

\section{Metodo di Gauss-Newton}
Vogliamo cercare di approssimare quindi le matrici hessiane con 
la parte del primo ordine, che domina la parte del secondo
ordine quando i residui sono piccoli.

\paragraph{Notazioni}
Cerchiamo delle notazioni comode per fare le nostre
considerazioni
$$ J_k = J_{R}(x^{k}), \quad R_k = R(x^k)= (r_1(x^{k}), \ldots,
r_n(x^{k}))
\quad
\text{ Funzioni residuali:$r_j$
}
$$
Il gradiente \`e definito nel seguente modo
$$ \nabla f(x^{k}) = J_{k}^{T}r_k,\quad$$
mentre la matrice hessiana viene approssimata
come segue
$$\nabla^{2}f(x^{k}) \approx J_k^{T}J_k $$
Il metodo di Gauss-Newton 
$$ d^{k}_{GN} \in \mathbb{R}^{n} 
\text{ soluzione di } \underbracket{J_{k}^{T}J_kd = - J_k^{T}r_k}_{\text{sistema lineare}}$$
Mentre nel metodo di Newton standard per avere una direzione di discesa
era necessario che la matrice hessiana fosse definita positiva,
qui invece le condizioni sono meno strette.
Vale infatti il seguente teorema:

\begin{theo}
Se $\nabla f(x^{k}) \neq 0$, allora $d_{GN}^{k}$ è una direzione
di discesa
\end{theo}
\begin{thproof}
Facciamo una serie di sostituzioni
$$ 
\begin{array}{l}
\nabla f(x^{k})^{T}d_{GN}^{k} = 
(J_k^{T}r_k)^{T} d_{GN}^{k}  \underbracket{=}_{\text{sistema lineare}} 
-(J_k^{T} J_{k} d_{GN}^{k})^{T} d_{GN}^{k} = \\
\underbracket{=}_{\text{reverse order law}} 
-(J_{k}^{T} d_{GN}^{k})^{T}(J_{k}d_{GN}^{k})
= - ||J_k^{T} d_{GN}^{k}||_{2}^{2} \leq 0 
\end{array}
$$
Affinch\'e si abbia una direzione di discesa,
la quantit\`a deve essere minore di zero.
Ma se
$$ J_{k}^{T} = d_{GN}^{k} = 0 \rightarrow
J_{k}^{T}r_k = 0 \text{ abbiamo } \nabla f(x^{k}) = 0 $$
Quindi quando il gradiente non \`e zero, abbiamo una
direzione di discesa.
\end{thproof}

\paragraph{Legame tra Gauss Newton e minimi quadrati}
$$J_{k}^{T}J_kd = - J_k^{T}r_k$$
inoltre è il sistema delle equazioni normali associate al sistema
$J_kd + r_k=0 $, ovvero la direzione di Newton $d_{GN}^{k}$ è la
soluzione di un problema di minimi quadrati lineare, risolve il
problema dei minimi quadrati lineari
$$ \min \left\{ \frac{1}{2} || J_k d + r_k ||_{2}^{2} \quad d \in \mathbb{R}^{n} \right\} $$
Trovare quindi una direzione di Gauss Newton, corrisponde
a risolvere un problema dei minimi quadrati,
e conosciamo tre metodi che non richiedono
il calcolo della matrice hessiana. \\ \\
Possiamo analizzare il metodo anche da un'altro punto di vista, calcoliamo lo sviluppo di Taylor del primo ordine di $r_j(x^{k}+ d) $:
$$ r_j(x^{k}+ d)  \approx r_j(x^{k}) + \nabla r_j(x^{k})^{T} d$$
$$ f(x^k + d) = 
\frac{1}{2} \displaystyle \sum_{j=1}^{n} r_j^{2}(x^{k} + d) \approx
 \frac{1}{2} \displaystyle \sum_{j=1}^{m}
\left[ r_j(x^{k}) + \nabla r_j(x^{k})^{T}d\right]^{2} = 
\frac{1}{2} || r_k + J_kd ||_{2}^{2} $$
Quindi $d_{GN}^{k}$ si ottiene minimizzando questa approssimazione di $f(x^k + d)$. \\ \\

Potremmo pensare passi unitari in stile Newton, ma in questo caso
non sarebbe garantita la convergenza del metodo: bisogna aggiungere
una ricerca inesatta del passo.

\paragraph{Metodo di Gauss-Netwon (``damped GN'')}
\begin{enumerate}
\item si sceglie un punto $x_0 \in \mathbb{R}^{n}$, $k=0$
\item se $\nabla f(x)=0$ (il punto trovato è stazionario),
      allora STOP.
\item altrimenti dobbiamo calcolare $d^{k}_{GN} \in \mathbb{R}^{n}$ soluzione
      di $J_{k}^{T}J_{k}d = -J_{k}r_{k}$
\item calcolare $t_k > 0 $ che soddisfi le condizioni di Wolfe
\item $x^{k+1} = x^{k} + t_k d_{GN}^{k}$
\item $k = k+1$ e tornare a 2)
\end{enumerate}

\begin{theo}[Teorema di convergenza]
Supponiamo che 
\begin{itemize}
\item $L_f(x^{0}) = \{ x \in \mathbb{R}^{n} : f(x) \leq f(x^{0}) \}$
     (curve di sottolivello) sia compatto
\item $r_j$ siano differenziabili con continuit\`a  per ogni $j=1\ldots n$
 \item $\nabla f$ sia continua di tipo Lipschitziano
\item $\exists \gamma > 0  \text{ t.c. } || J_kz||_{2} \geq \gamma ||z||_{2} 
\quad \forall z \in \mathbb{R}^{n} \; \forall k \text{ (iterazioni del metodo)} \quad (***)$ 
\end{itemize}
Allora 
$$ \lim_{k \to +\infty} || \nabla f(x^{k})||_{2} = 0 $$
\end{theo}
Vediamo il significato dell'ultima ipotesi:
ci garantisce che i valori singolari di $J_k$ siano uniformemente
distanti da 0.
Cerchiamo di capire perch\'e: sia
$ z=v_i $ vettore singolare destro di $J_k$: ha norma 1
$$ || J_k v_i||_{2} = \sigma_i \quad \sigma_i \geq \gamma \text{ per ogni
matrice } J_k $$
Quindi per ogni iterazione del metodo tutti i valori singolari saranno maggiori di $\gamma$.
\\
($\sigma_i$: valori singolari)
\begin{thproof}
\begin{todo}
Fare rivedere a Bigi la dimostrazione
\end{todo}
Per ipotesi $x \rightarrow J_r(x)$ è una funzione continua.
Allora $x \rightarrow || J_R(x)||_{2}$ continua, poich\'e
la norma \`e continua.
La curva di sottolivello $L_{f}(x^{0})$  per ipotesi \`e compatta.
Allora, per il teorema di Weirstrass esiste un massimo
$$ \exists \; \beta > 0 \quad 
|| J_{R}(x)||_{2} \leq \beta \quad \forall x \in L_f(x^{0}) 
$$
Sia $\theta_k$ l'angolo tra $-\nabla f(x^{k})$ e $d_{GN}^{k}$.
Abbiamo
$$ \cos \theta_{k} = 
\frac{-f(x^{k})^{T}d_{GN}^{k}}{|| \underbracket{\nabla f(x^{k})}_{(*)}||_{2} 
\; ||d_{GN}^{k}||_{2}} =
 \frac{|| J_{k}^{T}d_{GN}^{k} ||_{2}^{2} }{|| \underbracket{J_{k}^{T}J_k d_{GN}^{k}}_{(*)}||_{2}
||d_{GN}^{k}||_{2} } \geq \ldots (**)
$$
Valgono
\begin{itemize}
\item $ J_k^{T}r_k = -J_k^{T}J_{k}d_N^{k} \qquad (*)$
\item 
$
 || J_k^{T} J_k d_{GN}^{k}||_{2} \leq || J_{k} ||_{2}^{2}||d_{GN}^{k}||_{2}
\leq \beta^{2} ||d_{GN}^{k}||_{2} \qquad (**)
$
\end{itemize}
$$ \ldots \geq
\frac{|| J_k^{T} d_{GN}^{k}||_{2}^{2}}{\beta^{2} ||d^{k}_{GN}||_{2}^{2}}
\underbracket{\geq}_{(***)} \frac{\gamma^{2}||d_{GN}^{k}||_{2}^{2}}{\beta^{2} ||d_{GN}||{2}^{2}}$$
Dato Wolfe, Lipschitziana, funzione limite 
$$
\displaystyle \sum_{k=0}^{\infty} \underbracket{\cos^{2} \theta_k}_{\neq 0} || \nabla f(x^{k})||^{2} < + \infty \quad \Longrightarrow \quad \lim_{k \to + \infty} || \nabla f(x^{k})||_{2} = 0$$
\end{thproof}
In realt\`a  la condizione di Lipschitziana non è necessaria.


%% Da qui parte non revisionata
%% 29 Marzo 2011
\section{Regressioni (non) lineari - data/curve fitting}
Contesto: immaginiamo di avere un gruppo di dati, istanti di
tempo 
$$ y \in \mathbb{R}\; t_j \in \mathbb{R}^{p} $$
Vediamo se ci sono delle correleazioni tra i $t_j$
e gli $y$.
L'idea è avere una black box che prende in input
i $t_j$ e sputa fuori i $t_j$
$$ y \approx f(\underbracket{t_j}_{\text{ regressori }})$$
Per cercare di capire il comportamento della black box
possiamo fare
$$
\min\left\{ \frac{1}{2}
\left(\displaystyle \sum_{j=1}^{m} [y_j - f(t_j)]^{2} \right)
; \quad f \in T
\right\}
$$
Non abbiamo un vettore come incognita, abbiamo un insieme di funzioni.
In $\mathbb{R}^{n}$ abbiamo una base finita, nello spazio
di funzioni abbiamo una base infinita, le variabili sono
delle funzioni. Questo argomento non fa parte del corso.
Cerchiamo però di vedere un esempio.
$$ T = \{ f(x_1,\ldots, x_m; \cdot) :
\underbracket{x_1, x_2, \ldots, x_n}_{\text{parametri}} \in \mathbb{R}  \}$$
Assegnato il valore ad un insieme finito di parametri,
otteniamo la funzione corrispondente.
Abbiamo quindi ristretto l'insieme delle funzioni che è possibile
considerare, questo per un discorso di rappresentazione.
Basta riprendere l'esempio della prima lezione
di curve fitting per avere un esempio concreto.
Il nuovo problema è espresso come
$$ \min \left\{ \frac{1}{2} 
\displaystyle \sum_{j=1}^{n}[y_j - f(x_1,\ldots, x_n)]^{2}\; : \;
x_1, x_2, \ldots, x_n \in \mathbb{R} \right\}
$$
Esempio: Modello di crescita di un nuovo servizio.
si vuole ottenere un modello.
Vari modelli di crescita:
\begin{itemize}
\item Crescita esponenziale: $eg(t)  = x_1 e^{x_2 t}$
\item Potenza temporale: $tp(t)= x_1t^{x_2}$
 \item Crescita limitata $lmg(t) = x_3 - x_1^{-x_2 t}$
\item Logistica: $lgs(t) = x_3 / (1 + x_1 e^{-x_2 t})$
\end{itemize}
I primi due non hanno un bound superiore, le ultime due si.
Nell'esempio abbiamo gli ultimi 3 modelli ad essere
candidati.

$$
 \min \left\{ \frac{1}{2} 
\displaystyle \sum_{j=1}^{n}[y_j - tp(x_1,x_2;t_j)]^{2}\; : \;
x_1, x_2 \in \mathbb{R} \right\}
$$
$$ tp(x_1,x_2,t) = x_1 t^{x_2}$$
Nella lezione avevamo gli elementi residuali scritti come
$$ r_j(x_1, x_2) = y_j - x_1 t_j^{x_2}$$
Abbiamo 36 funzioni residuali candidate, e si può
utilizzare Gauss Newton per risolvere il problema.
\texttt{lsqnonlin} in Matlab.
Da notare che nella teoria, come criterio d'arresto
abbiamo
$$\nabla f(x^{*})=0$$
nella pratica si utilizza
\begin{itemize}
\item $ || \nabla f(x^{k})|| \leq \text{ tol } $
 \item $ |f(x^{k-1}) - f(x^{k})| \leq \text{ tol } $
\end{itemize}
Secondo caso
$$
 \min \left\{ \frac{1}{2} 
\displaystyle \sum_{j=1}^{n}[y_j - lmg(x_1,x_2;t_j)]^{2}\; : \;
x_1, x_2 , x_3 \in \mathbb{R} \right\}
$$

$$ tp(x_1,x_2, x_3, t) =x_3 - x_1 e^{-x_2 t}$$
Ultimo caso: funzione logistica

$$
 \min \left\{ \frac{1}{2} 
\displaystyle \sum_{j=1}^{n}[y_j - lmg(x_1,x_2;t_j)]^{2}\; : \;
x_1, x_2, x_3 \in \mathbb{R} \right\}
$$

$$lgs(x_1,x_2, x_3, t) =x_3 /(1+x_1e^{-x_2t})$$

\outbpdocument

